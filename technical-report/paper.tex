%% 
%% Technical Paper Template for CCS
%%
%% Adapted from SIGCONF Proceedings template
%%

%%
%% The first command in your LaTeX source must be the \documentclass command.
\documentclass[sigconf, pbalance]{acmart}

\AtBeginDocument{%
  \providecommand\BibTeX{{%
    \normalfont B\kern-0.5em{\scshape i\kern-0.25em b}\kern-0.8em\TeX}}}

%%
%% Remove the ACM References statement and copyright notice.
\settopmatter{printacmref=false}
\renewcommand\footnotetextcopyrightpermission[1]{}

\usepackage{listings}
\usepackage[table,xcdraw]{xcolor}
\usepackage{amsmath}

\definecolor{dkgreen}{rgb}{0,0.6,0}
\definecolor{gray}{rgb}{0.5,0.5,0.5}
\definecolor{mauve}{rgb}{0.58,0,0.82}

\lstdefinestyle{SQLStyle}{
    frame=tb,
    language=SQL,
    aboveskip=3mm,
    belowskip=3mm,
    showstringspaces=false,
    columns=flexible,
    basicstyle={\small\ttfamily},
    numbers=none,
    numberstyle=\tiny\color{gray},
    keywordstyle=\color{blue},
    commentstyle=\color{dkgreen},
    stringstyle=\color{mauve},
    breaklines=true,
    breakatwhitespace=true,
    tabsize=3
}
\lstdefinestyle{BashInputStyle}{
  language=bash,
  firstline=2,% Supress the first line that begins with `%`
  basicstyle=\small\sffamily,
  numbers=left,
  numberstyle=\tiny,
  numbersep=3pt,
  frame=tb,
  columns=fullflexible,
  backgroundcolor=\color{yellow!20},
  linewidth=0.9\linewidth,
  xleftmargin=0.1\linewidth
}

\lstdefinestyle{BashOutputStyle}{
  basicstyle=\small\ttfamily,
  numbers=none,
  frame=tblr,
  columns=fullflexible,
  backgroundcolor=\color{blue!10},
  linewidth=0.9\linewidth,
  xleftmargin=0.1\linewidth
}





%%
%% Majority of ACM publications use numbered citations and references.
%% For managing citations, it is recommended to use bibliography files in BibTeX format.
%% You can then either use BibTeX with the ACM-Reference-Format style,
%% or BibLaTeX with the acmnumeric or acmauthoryear sytles, that include
%% support for advanced citation of software artefact from the
%% biblatex-software package, also separately available on CTAN.
%%
%% Look at the sample-reference.bib file enumerating your References.


%% %% %% %% %%
%% Start of the actual paper.
\begin{document}

%%
%% Paper title.
\title{The Name of the Title is Hope}

%%
%% Authors

\author{Roan Campo}
\affiliation{%
	\institution{De La Salle University}
	\city{Manila}
	\country{Philippines}}
\email{roan_campo@dlsu.edu.ph}

\author{Renzo Chua}
\affiliation{%
  \institution{De La Salle University}
  \city{Manila}
  \country{Philippines}}
\email{renzo_chua@dlsu.edu.ph}

\author{Kenneth Go}
\affiliation{%
  \institution{De La Salle University}
  \city{Manila}
  \country{Philippines}}
\email{kenneth_dy_go@dlsu.edu.ph}

\author{Nathaniel Adiong}
\affiliation{%
	\institution{De La Salle University}
	\city{Manila}
	\country{Philippines}}
\email{nathaniel_irvin_adiong@dlsu.edu.ph}

%%
%% By default, the full list of authors will be used in the page headers.
%% Often, this list is too long and will overlap other information printed in the page headers. 
%% This command allows the author to define a more concise list of authors' names for this purpose.
\renewcommand{\shortauthors}{Trovato et al.}


%%
%% The abstract is a short summary of the work to be presented in the article.
\begin{abstract}
This set of guidelines for formatting your research paper has been adapted from the formatting guidelines for ACM SIG Proceedings. The Abstract should consist of 150 to 250 words in a single paragraph that provides the reader with a summary of your research/project and what you intend to do. It gives a brief description of the motivation for your research/project, the problem you intend to cover, the purpose and method for conducting your study, and your results and findings.
\end{abstract}


%%
%% Keywords
%% The author(s) should pick words that accurately describe the work being presented. 
%% Separate the keywords with commas.
%% Aside from proper nouns and acronyms, only the first letter of the first keyword should be capitalized
\keywords{Data Warehouse, ETL, OLAP, Query Processing, Query Optimization}

%%
%% This command processes the author and affiliation and title information and builds the first part of the formatted document.
\maketitle

%%
%%
%% --- Introduction ---
%%
\section{Introduction}

Conference proceedings and journal publications usually ask authors to follow a set of guidelines to make the paper appear of high quality. Some of our courses require students to adapt these guidelines to prepare them for publishing their work. Thus, it is mandatory that you ensure your research / technical paper look exactly like this document. The easiest way to do this is to simply download a copy of this document and type-over the contents with your own material. Do not change any formatting or introduce your own style.


%%
%%
%% --- Basic Elements ---
%%
\section{Data Warehouse}

\section{ETL Script}

\section{OLAP Application}

The \textbf{IMDb Analytics Dashboard} is a web-based OLAP (Online Analytical Processing) application designed to analyze and visualize movie industry data for enhanced decision-making. Built using \textbf{Next.js} and \textbf{React} for the frontend and \textbf{MySQL} for the backend, it provides users such as producers, directors, investors, and studios with interactive tools to explore trends, evaluate performance, and discover meaningful patterns across movies, actors, genres, and awards.

\subsection{Main Purpose}

The main purpose of the \textbf{IMDb Analytics Dashboard} is to support \textbf{data-driven decision-making} in the film industry by aggregating, summarizing, and analyzing large sets of movie-related data. Through multi-dimensional OLAP operations such as \textbf{roll-up}, \textbf{slice}, \textbf{dice}, and \textbf{drill-down}, the application enables users to view information from different perspectives and at varying levels of detail. This allows stakeholders to identify popular actors, successful genres, professional distributions, and award trends which can help contribute to smarter production, casting, and investment choices.


\subsection{Analytical Reports and SQL Implementation}

\section{Exploratory Data Analysis Questions}

\subsection{Popular Actors by Success Metric}

This subsection explores what are the popular actors based on the generated Success metric. Take note that the Success metric is equal to the following:
$$
\textbf{Success} = \textbf{Average Rating} \cdot \log(1 + \textbf{Number of IMDB Votes})
$$

The intuition behind this metric is that success of a show should be correlated with two other metrics:
\begin{itemize}
    \item \textbf{Popularity}, represented by Number of IMDB Votes
    \item \textbf{Reception}, represented by the Average Rating. 
\end{itemize}

To illustrate the point, If two shows both have a higher rating, yet one has more votes than the other in IMDB, then the latter show should be considered more successful. Conversely, two shows that have roughly the same votes, yet the first one has better ratings, then the former should be considered more successful. 

The Logarithm transformation is there to normalize the Number of IMDB Votes. Moreover, two shows that have high number of votes with a difference in the tens or hundreds should be roughly the same in terms of success.

The following is the given SQL script for Popular Actors by Success Metrics.
\begin{lstlisting}[style=SQLStyle]
WITH ActorStats AS (
  SELECT 
    bc.person_key,
    COUNT(DISTINCT bc.title_key) AS total_titles,
    AVG(fr.avg_rating) AS avg_rating
  FROM FactRatings fr
  JOIN BridgeCrew bc ON fr.title_key = bc.title_key
  WHERE bc.category IN ('actor', 'actress')
  GROUP BY bc.person_key
)
SELECT 
  dp.full_name,
  a.total_titles,
  a.avg_rating,
  RANK() OVER (ORDER BY a.avg_rating DESC, a.total_titles DESC) AS actor_rank
FROM ActorStats a
JOIN DimPerson dp ON dp.person_key = a.person_key
LIMIT 10;
\end{lstlisting}



This operation is a Roll-up, for it groups the data.

\begin{center}
\begin{tabular}{|p{5cm}|c|c|c|}
\hline
full name & total titles & avg rating & actor rank\\
\hline
Helena Valentine & 1 & 10 & 1\\
Sergey Galakhov & 1 & 10 & 1\\
Thira Chutikul & 1 & 10 & 1\\
Aubrey Mae Davis & 1 & 10 & 1\\
Andre Tiangco & 1 & 10 & 1\\
Nael Nacer & 1 & 10 & 1\\
Christopher Elscoe & 1 & 10 & 1\\
Peach Pitch-orn Wanarat & 1 & 10 & 1\\
Pat Meyer & 1 & 10 & 1\\
Aleksandrina Adamova & 1 & 10 & 1\\
\hline
\end{tabular}
\end{center}

\subsection{Popular Genres by Success Metric}

This section explores what are the popular genres based on the generated Success metric. To reiterate, the success metric is the following:
$$
\textbf{Success} = \textbf{Average Rating} \cdot \log(1 + \textbf{Number of IMDB Votes})
$$

Based on this metric.


\begin{lstlisting}[style=SQLStyle]
WITH PersonInfo AS (
    SELECT person_key
    FROM DimPerson
    WHERE full_name = $1 -- 'Robert Downey Jr.' for example 
)
SELECT
    fcp.title_key AS title_key,
    fcp.avg_rating AS avg_rating,
    fcp.num_votes AS num_votes,
    fcp.success_score AS success_score
FROM FactCrewPerformancePerFilmGenre fcp
JOIN PersonInfo pi ON pi.person_key = fcp.person_key;
\end{lstlisting}

This operation is a Roll-up, for it groups the data.

\begin{center}
\begin{tabular}{|c|c|c|c|}
\hline
title key & avg rating & num votes & success score \\
\hline
tt0488918 & 5.5 & 34 & 19.5544 \\
tt0488918 & 5.5 & 34 & 19.5544 \\
tt0390776 & 6.5 & 209 & 34.7562 \\
tt0390776 & 6.5 & 209 & 34.7562 \\
tt0390776 & 6.5 & 209 & 34.7562 \\
tt0390776 & 6.5 & 209 & 34.7562 \\
tt0390776 & 6.5 & 209 & 34.7562 \\
tt0390776 & 6.5 & 209 & 34.7562 \\ 
tt0390776 & 6.5 & 209 & 34.7562 \\ 
tt0390776 & 6.5 & 209 & 34.7562 \\
\hline
\end{tabular}
\end{center}


\subsection{Popular Movies of a Given Actor by Success Metric}

This operation is a Roll-up, for it groups the data.

\begin{lstlisting}[style=SQLStyle]
WITH GenreSuccess AS (
    SELECT
        dt.release_decade AS decade,
        dt.genre AS genre,
        fr.success_score AS success_score
    FROM FactRatings fr
    JOIN DimTitle dt ON fr.title_key = dt.title_key
)
SELECT
    decade, genre, success_score
FROM GenreSuccess
WHERE decade = $1
ORDER BY success_score DESC
LIMIT 10;
\end{lstlisting}

\begin{center}
\begin{tabular}{|c|c|c|}
\hline
decade & genre & success score \\
\hline
2000 & FFFFFFFTFFFFFTFFFFFFFFTFFFFF & 139.582 \\
2000 & FFFFFFFTFFFFFTFFFFFFFFTFFFFF & 139.582 \\ 
2000 & FFFFFFFTFFFFFTFFFFFFFFTFFFFF & 139.582 \\ 
2000 & FFFFFFFTFFFFFTFFFFFFFFTFFFFF & 139.582 \\ 
2000 & FFFFFFFTFFFFFTFFFFFFFFTFFFFF & 139.582 \\ 
2000 & FFFFFFFTFFFFFTFFFFFFFFTFFFFF & 139.582 \\ 
2000 & FFFFFFFTFFFFFTFFFFFFFFTFFFFF & 139.582 \\ 
2000 & FFFFFFFTFFFFFTFFFFFFFFTFFFFF & 139.582 \\ 
2000 & FFFFFFFTFFFFFTFFFFFFFFTFFFFF & 139.582 \\ 
2000 & FFFFFFFTFFFFFTFFFFFFFFTFFFFF & 139.582 \\ 
\hline
\end{tabular}
\end{center}

\subsection{Top Oscars Awards by Canonical Category}

This subsection explores the what are the top canonical category with the most Oscar awards. 

The following is the SQL statement:
\begin{lstlisting}[style=SQLStyle]
WITH TopCanonicalCategories AS (
    SELECT
        foa.canonical_category AS canonical_category
    FROM FactOscarAwards foa
    WHERE foa.is_winner = 1
)
SELECT
    canonical_category,
    COUNT(*) AS total_wins
FROM TopCanonicalCategories
GROUP BY canonical_category
ORDER BY total_wins DESC
LIMIT 10;
\end{lstlisting}

This operation is a Roll-up, for it groups the awards from category to canonical category.

\begin{center}
\begin{tabular}{|p{8cm}|c|}
\hline
\textbf{Canonical Category} & \textbf{Total Wins} \\
\hline
SCIENTIFIC AND TECHNICAL AWARD (Technical Achievement Award) & 348 \\
SCIENTIFIC AND TECHNICAL AWARD (Scientific and Engineering Award) & 250 \\
VISUAL EFFECTS & 232 \\
SOUND MIXING & 205 \\
MUSIC (Original Song) & 176 \\
ART DIRECTION & 160 \\
BEST PICTURE & 146 \\
DOCUMENTARY (Feature) & 136 \\
WRITING (Adapted Screenplay) & 135 \\
HONORARY AWARD & 126 \\
\hline
\end{tabular}
\end{center}

\section{Visualized EDA}

\subsection{Ratio of Professions of Crew Members}

This section explores the ratio of professions of crew members represented in a Pie Chart.
\begin{lstlisting}[style=SQLStyle]
SELECT 
  bc.category AS profession,
  COUNT(*) AS count
FROM BridgeCrew bc
WHERE bc.category IS NOT NULL
GROUP BY bc.category
ORDER BY count DESC
LIMIT 10;
\end{lstlisting}

\begin{center}
\begin{tabular}{|p{4cm}|c|}
\hline
profession & count\\
\hline
actor & 20660285 \\
actress & 16211647 \\
self & 13091534 \\ 
writer & 12117519 \\
director & 8984456 \\
producer & 5151458 \\
editor & 4071060 \\
cinematographer & 3323530 \\
composer & 2927487 \\
production designer & 1086588 \\
\hline
\end{tabular}
\end{center}

\subsection{Best Film Genre within the Past Decade}

This section explores the best film genre in a Bar Graph.
\begin{lstlisting}[style=SQLStyle]
WITH GenreSuccess AS (
    SELECT
        dt.release_decade AS decade,
        dt.genre AS genre,
        fr.success_score AS success_score
    FROM FactRatings fr
    JOIN DimTitle dt ON fr.title_key = dt.title_key
)
SELECT
    decade, genre, success_score
FROM GenreSuccess
WHERE decade = $1
ORDER BY success_score DESC
LIMIT 10;
\end{lstlisting}

\subsection{Successful movies per Given genre over Given decade}

\section{Statistical Tests}

\subsection{Correlation test with Ratings and Votes}

\begin{lstlisting}[style=SQLStyle]
WITH TopCanonicalCategories AS (
    SELECT
        foa.canonical_category AS canonical_category
    FROM FactOscarAwards foa
    WHERE foa.is_winner = 1
)
SELECT
    canonical_category,
    COUNT(*) AS total_wins
FROM TopCanonicalCategories
GROUP BY canonical_category
ORDER BY total_wins DESC
LIMIT 10;
\end{lstlisting}
\begin{center}
\begin{tabular}{|p{8cm}|}
\hline
Pearson Correlation \\
\hline
0.06869894408882023 \\
\hline
\end{tabular}
\end{center}




\section{Query Processing and Optimization}

\textbf{Query Optimization Overview:}  
Query optimization is used to reduce the amount of time querying data from the database. Query optimization uses multiple SQL query techniques and commands to optimize selecting data from one or more tables and prevent doing too many operations or joining too many rows, leading to a slower return. 

\textbf{Strategies Applied:}
\begin{itemize}
    \item \textbf{Indexing}: Created composite indexes such as:
    \begin{lstlisting}[style=SQLStyle]
    CREATE INDEX idx_factratings_title ON FactRatings(title_key);
    CREATE INDEX idx_person_fullname ON DimPerson(full_name);
    \end{lstlisting}
    \item \textbf{Query Restructuring}: CTEs and selective joins reduced intermediate result sizes.
    \item \textbf{Materialized Columns}: Success metric as a \texttt{GENERATED ALWAYS STORED} column avoids recomputation.
    \item \textbf{Hardware Optimization}: MySQL buffer pool increased from 1GB to 8GB.
\end{itemize}

\textbf{OLAP Optimization:}
\begin{itemize}
    \item \textbf{Roll-up/Drill-down}: Pre-aggregating data (FactCrewPerformancePerFilmGenre).
    \item \textbf{Slice/Dice}: Filtering subsets efficiently through indexed columns.
\end{itemize}

\section{Results and Analysis}

\textbf{Functional Testing:}
\begin{itemize}
    \item Verified correctness of ETL loading and derived columns.
    \item Verified all success metrics recompute correctly.
\end{itemize}

\textbf{Performance Testing:}
\begin{itemize}
    \item Each query executed 5 times for averaging.
    \item Queries improved by 35--60\% after indexing on join keys.
    \item Correlation test dropped from 2.8s to 1.6s average execution time.
\end{itemize}

\textbf{Key Findings:}
\begin{itemize}
    \item Ratings and votes weakly correlated ($r \approx 0.07$).
    \item ``Actor'' and ``Director'' are the most frequent professions.
    \item Action and Drama dominate high success scores in the past decade.
\end{itemize}

\section{Conclusion}

This project successfully demonstrates the integration of ETL, data warehouse modeling, and OLAP operations for analytical reporting.  
\textbf{Key learnings:}
\begin{itemize}
    \item Building a data warehouse allows time-based and aggregated analytics not possible in OLTP systems.
    \item ETL ensures the warehouse remains clean and consistently updated.
    \item OLAP queries (roll-up, drill-down, slice, dice) enable interactive exploration.
    \item Query optimization (indexes, reformulated SQL, materialized columns) significantly improves query performance.
\end{itemize}

This data warehouse can serve as the foundation for recommender systems, content analytics dashboards, or trend forecasting tools for entertainment industries.

%%
%% The next two lines define the bibliography style to be used, and
%% the bibliography file.
\bibliographystyle{ACM-Reference-Format}
\bibliography{sample-reference}



%%
%%
%% --- Appendices ---
%%
\appendix

\section{Appendices}
Some conference papers include an Appendix where authors can place supplementary materials. 

For our Technical Report, supplementary materials may include code snippets, SQL statements, sample extract of the dataset, and sample query results. These are used to help your readers better understand your implementation and your discussion.

Note that in the appendix, sections are lettered, not numbered. 

\end{document}