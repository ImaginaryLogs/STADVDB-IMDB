%% 
%% Technical Paper Template for CCS
%%
%% Adapted from SIGCONF Proceedings template
%%

%%
%% The first command in your LaTeX source must be the \documentclass command.
\documentclass[sigconf, pbalance]{acmart}

\AtBeginDocument{%
  \providecommand\BibTeX{{%
    \normalfont B\kern-0.5em{\scshape i\kern-0.25em b}\kern-0.8em\TeX}}}

%%
%% Remove the ACM References statement and copyright notice.
\settopmatter{printacmref=false}
\renewcommand\footnotetextcopyrightpermission[1]{}


%%
%% Majority of ACM publications use numbered citations and references.
%% For managing citations, it is recommended to use bibliography files in BibTeX format.
%% You can then either use BibTeX with the ACM-Reference-Format style,
%% or BibLaTeX with the acmnumeric or acmauthoryear sytles, that include
%% support for advanced citation of software artefact from the
%% biblatex-software package, also separately available on CTAN.
%%
%% Look at the sample-reference.bib file enumerating your References.


%% %% %% %% %%
%% Start of the actual paper.
\begin{document}

%%
%% Paper title.
\title{The Name of the Title is Hope}

%%
%% Authors

\author{Roan Campo}
\affiliation{%
	\institution{De La Salle University}
	\city{Manila}
	\country{Philippines}}
\email{roan_campo@dlsu.edu.ph}

\author{Renzo Chua}
\affiliation{%
  \institution{De La Salle University}
  \city{Manila}
  \country{Philippines}}
\email{renzo_chua@dlsu.edu.ph}

\author{Kenneth Go}
\affiliation{%
  \institution{De La Salle University}
  \city{Manila}
  \country{Philippines}}
\email{kenneth_dy_go@dlsu.edu.ph}

\author{Nathaniel Adiong}
\affiliation{%
	\institution{De La Salle University}
	\city{Manila}
	\country{Philippines}}
\email{nathaniel_irvin_adiong@dlsu.edu.ph}

%%
%% By default, the full list of authors will be used in the page headers.
%% Often, this list is too long and will overlap other information printed in the page headers. 
%% This command allows the author to define a more concise list of authors' names for this purpose.
\renewcommand{\shortauthors}{Trovato et al.}


%%
%% The abstract is a short summary of the work to be presented in the article.
\begin{abstract}
This set of guidelines for formatting your research paper has been adapted from the formatting guidelines for ACM SIG Proceedings. The Abstract should consist of 150 to 250 words in a single paragraph that provides the reader with a summary of your research/project and what you intend to do. It gives a brief description of the motivation for your research/project, the problem you intend to cover, the purpose and method for conducting your study, and your results and findings.
\end{abstract}


%%
%% Keywords
%% The author(s) should pick words that accurately describe the work being presented. 
%% Separate the keywords with commas.
%% Aside from proper nouns and acronyms, only the first letter of the first keyword should be capitalized
\keywords{Data Warehouse, ETL, OLAP, Query Processing, Query Optimization}

%%
%% This command processes the author and affiliation and title information and builds the first part of the formatted document.
\maketitle

%%
%%
%% --- Introduction ---
%%
\section{Introduction}

Conference proceedings and journal publications usually ask authors to follow a set of guidelines to make the paper appear of high quality. Some of our courses require students to adapt these guidelines to prepare them for publishing their work. Thus, it is mandatory that you ensure your research / technical paper look exactly like this document. The easiest way to do this is to simply download a copy of this document and type-over the contents with your own material. Do not change any formatting or introduce your own style.


%%
%%
%% --- Basic Elements ---
%%
\section{Data Warehouse}

\section{ETL Script}

\section{OLAP Application}

The \textbf{IMDb Analytics Dashboard} is a web-based OLAP (Online Analytical Processing) application that gives clear insights into movie industry trends, performance, and statistics. Built with Next.js and React for the frontend and MySQL for the backend, the dashboard helps producers, directors, investors, and studios make smarter, data-based decisions. Its main goal is to support data-driven decision-making by showing useful analytics on movie performance, crew success, audience engagement, and statistical trends.

\subsection{Core Analytical Features}

\subsubsection{Basic Statistics}

The dashboard presents key descriptive analytics through  visualizations, including:

\begin{itemize}
    \item \textbf{Genre Distribution} – Shows the proportion of movies across different genres.
    \item \textbf{Ratings Trend by Year} – Tracks how movie ratings have evolved over time.
    \item \textbf{Top Rated Movies} – Highlights the highest-rated films within selected periods.
    \item \textbf{Awards by Category} – Displays counts and trends of award nominations and wins per category.
    \item \textbf{Popular Actors} – Ranks actors based on audience engagement and average ratings.
    \item \textbf{Successful Crew Members} – Identifies top-performing directors, writers, and producers.
    \item \textbf{Best Film Genre (Past Decade)} – Determines which genre has performed best over the last ten years.
    \item \textbf{Ratio of Actor Professions} – Compares the distribution of professions (e.g., actors, directors, writers).
    \item \textbf{Popularity of Actors per Genre} – Examines which actors perform best within specific genres.
\end{itemize}

\subsubsection{Statistical and Inferential Analysis}

Beyond descriptive analytics, the dashboard incorporates  statistical techniques for deeper insight:

\begin{itemize}
    \item \textbf{Correlation Analysis (Pearson):} Measures the relationship between movie ratings and the number of audience votes, identifying whether higher popularity correlates with higher ratings.
    \item \textbf{Chi-Square Test:} Evaluates the association between movie genre and award wins, determining whether award distribution is genre-dependent.
    \item \textbf{Hypothesis Testing (T-Test):} Compares the average genre ratings against the overall population mean, assessing whether specific genres significantly outperform or underperform the industry average.
\end{itemize}

\subsection{Decision-Making and Analytical Tasks}

The \textbf{IMDb Analytics Dashboard} supports multiple decision-making processes, including:

\begin{enumerate}
    \item \textbf{Genre Popularity Analysis:} Determines top-performing genres that can be used for strategic production planning.
    \item \textbf{Crew Performance Evaluation:} Assesses directors, writers, and actors based on historical performance metrics.
    \item \textbf{Trend Analysis:} Examines rating and production changes over time to identify optimal release periods.
    \item \textbf{Award Prediction and Analysis:} Studies nomination and win trends to refine awards campaign strategies.
    \item \textbf{Actor-Genre Affinity:} Analyzes actor success within genres to guide casting and marketing decisions.
\end{enumerate}


\subsection{Analytical Reports and SQL Implementation}

\section{Query Processing and Optimization}

\section{Results and Analysis}

The sections for your Technical Report are described in the corresponding project specifications found in AnimoSpace. 
To use this ACM paper template, replace each section with the required section heading as indicated in the project specifications. For example, for MCO1, the sections are:
\begin{verbatim}
    1. Introduction
    2. Data Warehouse
    3. ETL Script
    4. OLAP Application
    5. Query Processing and Optimization
    6. Results and Analysis
    7. Conclusion
    8. References
    9. Declarations
\end{verbatim}

You may optionally include an Appendix to show your code snippets, SQL statements, sample extract of the dataset, and sample query results.



%%
%% The next two lines define the bibliography style to be used, and
%% the bibliography file.
\bibliographystyle{ACM-Reference-Format}
\bibliography{sample-reference}



%%
%%
%% --- Appendices ---
%%
\appendix

\section{Appendices}
Some conference papers include an Appendix where authors can place supplementary materials. 

For our Technical Report, supplementary materials may include code snippets, SQL statements, sample extract of the dataset, and sample query results. These are used to help your readers better understand your implementation and your discussion.

Note that in the appendix, sections are lettered, not
numbered. 

\end{document}
